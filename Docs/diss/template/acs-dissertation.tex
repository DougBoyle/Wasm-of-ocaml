\documentclass[a4paper,12pt,twoside,openright]{report}


\def\authorname{Douglas Boyle\xspace}
\def\authorcollege{Churchill College\xspace}
%\def\authoremail{Sarah.Jones@cl.cam.ac.uk}
\def\dissertationtitle{Optimising compiler from OCaml to Webassembly}
\def\wordcount{14,235}


\usepackage{epsfig,parskip,setspace,tabularx,xspace}

\usepackage[pdfborder={0 0 0}]{hyperref}    % turns references into hyperlinks
\usepackage[margin=25mm]{geometry}  % adjusts page layout
\usepackage{graphicx}  % allows inclusion of PDF, PNG and JPG images
\usepackage{verbatim}
\usepackage{docmute}   % only needed to allow inclusion of proposal.tex

%% START OF DOCUMENT
\begin{document}

%% FRONTMATTER (TITLE PAGE, DECLARATION, ABSTRACT, ETC)
%\pagestyle{empty}
%\singlespacing
%\input{titlepage}
%\onehalfspacing
%\input{declaration}
%\singlespacing
%\input{abstract}

%\pagenumbering{roman}
%\setcounter{page}{0}
%\pagestyle{plain}
%\tableofcontents
%\listoffigures
%\listoftables

%\onehalfspacing

\newpage
%%%%%%%%%%%%%%%%%%%%%%%%%%%%%%%%%%%%%%%%%%%%%%%%%%%%%%%%%%%%%%%%%%%%%%%%
% Title


\pagestyle{empty}

\rightline{\LARGE \textbf{\authorname}}

\vspace*{60mm}
\begin{center}
\Huge
\textbf{\dissertationtitle} \\[5mm]
Computer Science Tripos -- Part II \\[5mm]
\authorcollege \\[5mm]
\today  % today's date
\end{center}

%%%%%%%%%%%%%%%%%%%%%%%%%%%%%%%%%%%%%%%%%%%%%%%%%%%%%%%%%%%%%%%%%%%%%%%%%%%%%%
% Proforma, table of contents and list of figures
\pagestyle{plain}

\chapter*{Proforma}

{\large
\begin{tabular}{ll}
Name:               & \bf\authorname     \\
College:            & \bf\authorcollege     \\
Project Title:      & \bf\dissertationtitle \\
Examination:        & \bf Computer Science Tripos -- Part II, July 2021  \\
Word Count:         & \bf \wordcount \footnotemark[1]
                      (well less than the 12000 limit)  \\
Project Originator: & Dr T.~Jones                    \\
Supervisor:         &  Dr T.~Jones                   \\ 
\end{tabular}
}
\footnotetext[1]{This word count was computed
by \texttt{detex diss.tex | tr -cd '0-9A-Za-z $\tt\backslash$n' | wc -w}
}
\stepcounter{footnote}

\section*{Original Aims of the Project}

To write a demonstration dissertation\footnote{A normal footnote without the
complication of being in a table.} using \LaTeX\ to save
student's time when writing their own dissertations. The dissertation
should illustrate how to use the more common \LaTeX\ constructs. It
should include pictures and diagrams to show how these can be
incorporated into the dissertation.  It should contain the entire
\LaTeX\ source of the dissertation and the makefile.  It should
explain how to construct an MSDOS disk of the dissertation in
Postscript format that can be used by the book shop for printing, and,
finally, it should have the prescribed layout and format of a diploma
dissertation.


\section*{Work Completed}

All that has been completed appears in this dissertation.

\section*{Special Difficulties}

Learning how to incorporate encapulated postscript into a \LaTeX\
document on both Ubuntu Linux and OS X.
 
\newpage
\section*{Declaration}

I, \authorname of \authorcollege, being a candidate for Part II of the Computer
Science Tripos, hereby declare
that this dissertation and the work described in it are my own work,
unaided except as may be specified below, and that the dissertation
does not contain material that has already been used to any substantial
extent for a comparable purpose.

\bigskip
\leftline{Signed [signature]}

\medskip
\leftline{Date [date]}

%% START OF MAIN TEXT

\chapter{Introduction}

This is the introduction where you should introduce your work.  In
general the thing to aim for here is to describe a little bit of the
context for your work --- why did you do it (motivation), what was the
hoped-for outcome (aims) --- as well as trying to give a brief
overview of what you actually did.

It's often useful to bring forward some ``highlights'' into
this chapter (e.g.\ some particularly compelling results, or
a particularly interesting finding).

It's also traditional to give an outline of the rest of the
document, although without care this can appear formulaic
and tedious. Your call.


\chapter{Background}

A more extensive coverage of what's required to understand your
work. In general you should assume the reader has a good undergraduate
degree in computer science, but is not necessarily an expert in
the particular area you've been working on. Hence this chapter
may need to summarize some ``text book'' material.

This is not something you'd normally require in an academic paper,
and it may not be appropriate for your particular circumstances.
Indeed, in some cases it's possible to cover all of the ``background''
material either in the introduction or at appropriate places in
the rest of the dissertation.


\chapter{Related Work}

This chapter covers relevant (and typically, recent) research
which you build upon (or improve upon). There are two complementary
goals for this chapter:
\begin{enumerate}
  \item to show that you know and understand the state of the art; and
  \item to put your work in context
\end{enumerate}

Ideally you can tackle both together by providing a critique of
related work, and describing what is insufficient (and how you do
better!)

The related work chapter should usually come either near the front or
near the back of the dissertation. The advantage of the former is that
you get to build the argument for why your work is important before
presenting your solution(s) in later chapters; the advantage of the
latter is that don't have to forward reference to your solution too
much. The correct choice will depend on what you're writing up, and
your own personal preference.



\chapter{Design and Implementation}

This chapter may be called something else\ldots but in general
the idea is that you have one (or a few) ``meat'' chapters which
describe the work you did in technical detail.


\chapter{Evaluation}

For any practical projects, you should almost certainly have
some kind of evaluation, and it's often useful to separate
this out into its own chapter.


\chapter{Summary and Conclusions}

As you might imagine: summarizes the dissertation, and draws
any conclusions. Depending on the length of your work, and
how well you write, you may not need a summary here.

You will generally want to draw some conclusions, and point
to potential future work.




\appendix
\singlespacing

\bibliographystyle{unsrt}
%\bibliography{dissertation}

\end{document}
