\documentclass[a4paper,12pt,twoside]{report} % openright


\def\authorname{Douglas Boyle\xspace}
\def\authorcollege{Churchill College\xspace}
%\def\authoremail{Sarah.Jones@cl.cam.ac.uk}
\def\dissertationtitle{Optimising compiler from OCaml to WebAssembly}
\def\wordcount{\#\#\#\#\#}


\usepackage{epsfig,parskip,setspace,tabularx,xspace}

\usepackage[pdfborder={0 0 0}]{hyperref}    % turns references into hyperlinks
\usepackage[margin=25mm]{geometry}  % adjusts page layout
\usepackage{graphicx}  % allows inclusion of PDF, PNG and JPG images
\usepackage{verbatim}
\usepackage{docmute}   % only needed to allow inclusion of proposal.tex
\usepackage{fancyhdr}


%\renewcommand{\headrulewidth}{0.4pt}
%\fancyheadoffset[LO,LE,RO,RE]{0pt}
%\fancyfootoffset[LO,LE,RO,RE]{0pt}
%\pagestyle{fancy}
%\fancyhead{}
%\fancyhead[LO,RE]{{\bfseries \authorname}\\ dab80@cam.ac.uk}
%\setlength{\headheight}{22.55pt}



%% START OF DOCUMENT
\begin{document}

%% FRONTMATTER (TITLE PAGE, DECLARATION, ABSTRACT, ETC)
%\pagestyle{empty}
%\singlespacing
%% title page information
\begin{titlepage}

\begin{center}
\noindent
\huge
\dissertationtitle \\
\vspace*{\stretch{1}}
\end{center}

\begin{center}
\noindent
\huge
\authorname \\
\Large
\authorcollege      \\[24pt]
\includegraphics{CUni3.pdf}
\end{center}

\vspace{24pt}

\begin{center}
\noindent
\large
{\it A dissertation submitted to the University of Cambridge \\
in partial fulfilment of the requirements for the degree of \\
Master of Philosophy in Advanced Computer Science}
\vspace*{\stretch{1}}
\end{center}

\begin{center}
\noindent
University of Cambridge \\
Department of Computer
Science and Technology  \\
William Gates Building  \\
15 JJ Thomson Avenue    \\
Cambridge CB3 0FD       \\
{\sc United Kingdom}    \\
\end{center}

\begin{center}
\noindent
Email: \authoremail \\
\end{center}

\begin{center}
\noindent
\today
\end{center}

\end{titlepage}

\newpage
\vspace*{\fill}

%\onehalfspacing
%\newpage
{\Huge \bf Declaration}

\vspace{24pt}

I \authorname of \authorcollege, being a candidate for the M.Phil in
Advanced Computer Science, hereby declare that this report and the
work described in it are my own work, unaided except as may be
specified below, and that the report does not contain material that
has already been used to any substantial extent for a comparable
purpose.

\vspace{24pt}
Total word count: \wordcount

\vspace{60pt}
\textbf{Signed}:

\vspace{12pt}
\textbf{Date}:


\vfill

This dissertation is copyright \copyright 2010 \authorname.
\\
All trademarks used in this dissertation are hereby acknowledged.



\newpage
\vspace*{\fill}

%\singlespacing
%\newpage
{\Huge \bf Abstract}
\vspace{24pt}


This is the abstract. Write a summary of the whole thing. Make
sure it fits in one page.


\newpage
\vspace*{\fill}


%\pagenumbering{roman}
%\setcounter{page}{0}
%\pagestyle{plain}
%\tableofcontents
%\listoffigures
%\listoftables

%\onehalfspacing

 %\newpage
%%%%%%%%%%%%%%%%%%%%%%%%%%%%%%%%%%%%%%%%%%%%%%%%%%%%%%%%%%%%%%%%%%%%%%%%
% Title


%\pagestyle{empty}

%\rightline{\LARGE \textbf{\authorname} }

%\vspace*{60mm}
%\begin{center}
%\Huge
%\textbf{Progress Report} \\[5mm]
%Computer Science Tripos -- Part II \\[5mm]
%\today  % today's date
%\end{center}


%\newpage
%%%%%%%%%%%%%%%%%%%%%%%%%%%%%%%%%%%%%%%%%%%%%%%%%%%%%%%%%%%%%%%%%%%%%%%%%%%%%%
% Proforma, table of contents and list of figures
%\pagestyle{plain}


{\large
\begin{tabular}{ll}
Name:               & \authorname     \\
Email: 		& {dab80@cam.ac.uk} \\
%College:            & \authorcollege     \\
Project Title:      & \dissertationtitle \\
Supervisor:         &  Dr T.~Jones                   \\ 
Director of Studies: & Dr J.~Fawcett \\
Overseers: & Dr S.~Holden and Dr N.~Krishnaswami
\end{tabular}
}

Using the output of the OCaml compiler's parser and type-checker, I have implemented translation to a linearised intermediate representation, followed by WebAssembly code generation. I have also written a runtime system in WebAssembly providing functions such as memory allocation, and a small JavaScript wrapper to simplify interacting with the WebAssembly representations of values. Except for the module and object language and exceptions, these support all of the syntax constructs of OCaml. Rather than try to implement all of OCaml's primitive types and operations, just the integer, comparison, boolean and list operations were implemented initially. This has since been extended to support references and the basic floating point operations, allowing a wider range of programs to be compiled. 

I have also written a set of benchmark OCaml programs with equivalent programs in both C and Grain.
Where possible, these benchmarks were adapted from existing benchmarks available online so that they closer resemble real programs.
I have collected data on the performance of compiled code for each of these languages in terms of heap usage, execution time and file size. I additionally compared this with the result of running Js\_of\_ocaml on the OCaml benchmark programs and executing them in JavaScript.

% Include a graph of comparative performance?

This meets the original success criteria for my project. Having achieved that, I have implemented several optimisations to my compiler. These include tail call optimisations for both recursive and mutually recursive functions, constant propagation, and dead assignment elimination. All of these were done on the intermediate representation. In addition, a couple of optimisations have been implemented on the lower level representation. This representation maps directly to blocks and instructions in WebAssembly but also maintains a flow graph of possible execution sequences for easier analysis. Optimisations were identified by comparing compiled WebAssembly with the output of running the \verb|wasm-opt| tool on it. For example, dead assignments would sometimes be introduced for the temporary local variables used to compile IR operations.

% Say about measuring impact of optimisations?

Lastly, based on the runtime system of the Grain compiler, I have implemented a reference counting garbage collector in JavaScript which is imported into WebAssembly modules at runtime. This puts me about three weeks ahead of schedule. For the next few weeks before I start writing the dissertation, I will look at reducing the overhead of garbage collection and alternative methods capable of collecting cyclic structures. I also need to collect more detailed data on the performance impact of my optimisations and the benefit and overhead of garbage collection.

 
\end{document}
