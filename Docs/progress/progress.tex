\documentclass[a4paper,12pt,twoside]{report} % openright


\def\authorname{Douglas Boyle\xspace}
\def\authorcollege{Churchill College\xspace}
%\def\authoremail{Sarah.Jones@cl.cam.ac.uk}
\def\dissertationtitle{Optimising compiler from OCaml to Webassembly}
\def\wordcount{\#\#\#\#\#}


\usepackage{epsfig,parskip,setspace,tabularx,xspace}

\usepackage[pdfborder={0 0 0}]{hyperref}    % turns references into hyperlinks
\usepackage[margin=25mm]{geometry}  % adjusts page layout
\usepackage{graphicx}  % allows inclusion of PDF, PNG and JPG images
\usepackage{verbatim}
\usepackage{docmute}   % only needed to allow inclusion of proposal.tex


%% START OF DOCUMENT
\begin{document}

%% FRONTMATTER (TITLE PAGE, DECLARATION, ABSTRACT, ETC)
%\pagestyle{empty}
%\singlespacing
%\input{titlepage}
%\onehalfspacing
%\input{declaration}
%\singlespacing
%\input{abstract}

%\pagenumbering{roman}
%\setcounter{page}{0}
%\pagestyle{plain}
%\tableofcontents
%\listoffigures
%\listoftables

%\onehalfspacing

\newpage
%%%%%%%%%%%%%%%%%%%%%%%%%%%%%%%%%%%%%%%%%%%%%%%%%%%%%%%%%%%%%%%%%%%%%%%%
% Title


\pagestyle{empty}

\rightline{\LARGE \textbf{\authorname} }

\vspace*{60mm}
\begin{center}
\Huge
\textbf{Progress Report} \\[5mm]
Computer Science Tripos -- Part II \\[5mm]
\today  % today's date
\end{center}


\newpage
%%%%%%%%%%%%%%%%%%%%%%%%%%%%%%%%%%%%%%%%%%%%%%%%%%%%%%%%%%%%%%%%%%%%%%%%%%%%%%
% Proforma, table of contents and list of figures
\pagestyle{plain}


{\large
\begin{tabular}{ll}
Name:               & \bf\authorname     \\
Email: 		& \bf{dab80@cam.ac.uk} \\
College:            & \bf\authorcollege     \\
Project Title:      & \bf\dissertationtitle \\
Supervisor:         &  Dr T.~Jones                   \\ 
Director of Studies: & Dr J.~Fawcett \\
Overseers: & Dr S.~Holden and Dr N.~Krishnaswami
\end{tabular}
}


\section*{Work Completed} % Of course update this as the project develops
Using the parser and type-checker of the official OCaml compiler, I have implemented translation to a linearised IR and WebAssembly code generation from this. I have also written a runtime system in WebAssembly for performing memory allocation and comparisons, and a small JavaScript wrapper to simplify interacting with the WebAssembly representations of values. Except for the module and object language and exceptions, these support all of the syntax constructs of OCaml such as arbitrary pattern matching and polymorphic functions. Rather than try to implement all of OCaml's primitive operations, just the integer operations were implemented initially and this has been extended to support references and the basic floating point operations, allowing a wider range of programs to be compiled. 

I also wrote a set of benchmark OCaml programs with equivalent programs in both C and Grain, a functional language with a compiler to WebAssembly.
Where possible, these benchmarks were adapted from existing benchmarks found online so that they would closer resemble real programs.
I have collected data on the performance of compiled code for each of these languages in terms of heap usage, execution time and file size. I additionally compared this with the result of running Js\_of\_ocaml on the benchmark programs and executing them in JavaScript.

% Include a graph of comparative performance?

This meets the original success criteria for my project. Having achieved that, I have implemented several optimisations to my compiler. This includes tail call optimisations for both recursive and mutually recursive functions, constant propagation and dead assignment elimination. All of these were done on the intermediate representation. In addition, a small number of optimisations have been implemented on the WebAssembly representation of programs, which maps directly to blocks and instructions in WebAssembly but with instructions within a function linked together by a graph indicating possible execution orders. Optimisations were identified by comparing compiled WebAssembly with the output of running the \verb|wasm-opt| tool on it. This identified several peephole optimisations such as removing dead assignments introduced by the temporary local variables used to compile some of the IR operations.

% Say about measuring impact of optimisations?

I am therefore about a fortnight ahead of my schedule, so now plan to research implementing garbage collection in WebAssembly and decide if this is a worthwhile change to my compiler. If not, I will continue to further optimise my compiler, most likely by adding function inlining or optimising how functions are called where the function being called can be identified statically. After this I will begin writing the dissertation.

 
\end{document}
